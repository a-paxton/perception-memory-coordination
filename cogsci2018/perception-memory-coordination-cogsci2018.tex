% Template for Cogsci submission with R Markdown

% Stuff changed from original Markdown PLOS Template
\documentclass[10pt, letterpaper]{article}

\usepackage{cogsci}
\usepackage{pslatex}
\usepackage{float}
\usepackage{caption}

% amsmath package, useful for mathematical formulas
\usepackage{amsmath}

% amssymb package, useful for mathematical symbols
\usepackage{amssymb}

% hyperref package, useful for hyperlinks
\usepackage{hyperref}

% graphicx package, useful for including eps and pdf graphics
% include graphics with the command \includegraphics
\usepackage{graphicx}

% Sweave(-like)
\usepackage{fancyvrb}
\DefineVerbatimEnvironment{Sinput}{Verbatim}{fontshape=sl}
\DefineVerbatimEnvironment{Soutput}{Verbatim}{}
\DefineVerbatimEnvironment{Scode}{Verbatim}{fontshape=sl}
\newenvironment{Schunk}{}{}
\DefineVerbatimEnvironment{Code}{Verbatim}{}
\DefineVerbatimEnvironment{CodeInput}{Verbatim}{fontshape=sl}
\DefineVerbatimEnvironment{CodeOutput}{Verbatim}{}
\newenvironment{CodeChunk}{}{}

% cite package, to clean up citations in the main text. Do not remove.
\usepackage{cite}

\usepackage{color}

% Use doublespacing - comment out for single spacing
%\usepackage{setspace}
%\doublespacing


% % Text layout
% \topmargin 0.0cm
% \oddsidemargin 0.5cm
% \evensidemargin 0.5cm
% \textwidth 16cm
% \textheight 21cm

\title{Perception, Memory, and Coordination}


\author{{\large \bf Alexandra Paxton} \\ \texttt{paxton.alexandra@gmail.com} \\ Institute of Cognitive and Brain Sciences \\ Berkeley Institute for Data Science \\ University of California, Berkeley \And {\large \bf Thomas J. H. Morgan} \\ \texttt{thomas.j.h.morgan@asu.edu} \\ School of Human Evolution and Social Change \\ Arizona State University \AND {\large \bf Jordan W. Suchow} \\ \texttt{suchow@berkeley.edu} \\ Social Science Matrix \\ University of California, Berkeley \And {\large \bf Thomas L. Griffiths} \\ \texttt{tom\_griffiths@berkeley.edu} \\ Department of Psychology \\ University of California, Berkeley}

\begin{document}

\maketitle

\begin{abstract}
With cognitive scientists' increasing interest in moving outside of the
lab, recent advances in crowdsourcing platforms can help strike a
balance between the tight experimental control of lab designs and the
affordances of web-based experiments to reach beyond traditional
undergraduate subject pools. By taking advantage of new tools,
scientists interested in social cognition and behavior can create new
designs and adapt traditional ones to deliver experiments at scale.
Dallinger is one such tool, providing researchers with an open-source
experiment platform that provides end-to-end automation of the
experiment pipeline, from participant recruitment and consent to data
de-identification and participant compensation. Here we demonstrate how
Dallinger can be used to run complex experimental studies of interactive
human social behavior, as a demonstration of its potential to study
social cognition and behavior using designs drawn from across cognitive
science.

\textbf{Keywords:}
interpersonal interaction; human communication; crowdsourcing; Dallinger
\end{abstract}

\section{Introduction}\label{introduction}

\section{Method}\label{method}

All research activities were completed in compliance with oversight from
Committee for the Protection of Human Subjects at the University of
California, Berkeley.

\subsection{Participants}\label{participants}

Participants (\emph{n} = 12) were individually recruited from Amazon
Mechanical Turk to participate as dyads (\emph{n} = 6). Participants
were paired with one another according to the order in which they began
the experiment. All participants were over 18 years of age and fluent
English speakers (self-reported), located within the U.S.

The experiment lasted an average of 11.96 minutes (range: 8.13---17.66
minutes). In return for their participation, all participants were paid
\$1.33 simply for finishing the experiment. Each participant also earned
a bonus based on mean performance of up to \$2 (for the entire
experiment).

\textbf{NB}: Should probably report mean and range for bonus
here\ldots{}

\subsection{Procedure}\label{procedure}

All data collection procedures occurred on Dallinger
(\url{http://github.com/dallinger/Dallinger}), deployed on Amazon
Mechanical Turk (\url{http://mturk.com}). Code for the experiment is
available on GitHub:
\url{http://github.com/thomasmorgan/joint-estimation-game}, and the
resulting experiment data are available on the OSF repository for the
project: \url{https://osf.io/8fu7x/}.

Each participant was individually recruited on Amazon Mechanical Turk to
play a ``Line Estimation Game'' advertised as a way to ``Test your
memory skills!''.

Participants were given no information about their partner other than
the guess that their partner made.

\subsection{Analyses}\label{analyses}

\section{Results}\label{results}

\section{Discussion}\label{discussion}

\section{Conclusion}\label{conclusion}

\section{Acknowledgements}\label{acknowledgements}

Place acknowledgments (including funding information) in a section at
the end of the paper.

\section{References}\label{references}

\setlength{\parindent}{-0.1in} \setlength{\leftskip}{0.125in} \noindent

\end{document}
