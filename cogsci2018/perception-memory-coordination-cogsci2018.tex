% Template for Cogsci submission with R Markdown

% Stuff changed from original Markdown PLOS Template
\documentclass[10pt, letterpaper]{article}

\usepackage{cogsci}
\usepackage{pslatex}
\usepackage{float}
\usepackage{caption}

% amsmath package, useful for mathematical formulas
\usepackage{amsmath}

% amssymb package, useful for mathematical symbols
\usepackage{amssymb}

% hyperref package, useful for hyperlinks
\usepackage{hyperref}

% graphicx package, useful for including eps and pdf graphics
% include graphics with the command \includegraphics
\usepackage{graphicx}

% Sweave(-like)
\usepackage{fancyvrb}
\DefineVerbatimEnvironment{Sinput}{Verbatim}{fontshape=sl}
\DefineVerbatimEnvironment{Soutput}{Verbatim}{}
\DefineVerbatimEnvironment{Scode}{Verbatim}{fontshape=sl}
\newenvironment{Schunk}{}{}
\DefineVerbatimEnvironment{Code}{Verbatim}{}
\DefineVerbatimEnvironment{CodeInput}{Verbatim}{fontshape=sl}
\DefineVerbatimEnvironment{CodeOutput}{Verbatim}{}
\newenvironment{CodeChunk}{}{}

% cite package, to clean up citations in the main text. Do not remove.
\usepackage{cite}

\usepackage{color}

% Use doublespacing - comment out for single spacing
%\usepackage{setspace}
%\doublespacing


% % Text layout
% \topmargin 0.0cm
% \oddsidemargin 0.5cm
% \evensidemargin 0.5cm
% \textwidth 16cm
% \textheight 21cm

\title{Interpersonal Coordination of Perception and Memory\\in Real-Time Online
Social Experiments}


\author{{\large \bf Alexandra Paxton} \\ \texttt{paxton.alexandra@gmail.com} \\ Institute of Cognitive and Brain Sciences \\ Berkeley Institute for Data Science \\ University of California, Berkeley \And {\large \bf Thomas J. H. Morgan} \\ \texttt{thomas.j.h.morgan@asu.edu} \\ School of Human Evolution and Social Change \\ Arizona State University \AND {\large \bf Jordan W. Suchow} \\ \texttt{suchow@berkeley.edu} \\ Social Science Matrix \\ University of California, Berkeley \And {\large \bf Thomas L. Griffiths} \\ \texttt{tom\_griffiths@berkeley.edu} \\ Department of Psychology \\ University of California, Berkeley}

\begin{document}

\maketitle

\begin{abstract}
The quiet hum of interpersonal coordination that runs throughout social
communication and interaction shows how individuals can subtly influence
one another's behaviors, thoughts, and emotions over time. While the
majority of research on coordination studies face-to-face interaction,
recent advances in crowdsourcing afford the opportunity to conduct
large-scale, real-time social interaction experiments. We take advantage
of these tools to explore interpersonal coordination in a ``minimally
interactive context,'' distilling the richness of natural communication
into a tightly controlled setting to explore how people become coupled
in their perceptual and memory systems while performing a task together.
Consistent with previous work on postural sway and gaze, we found that
individuals become coupled to one another's cognitive processes without
needing to be co-located or fully interactive with their partner;
interestingly, although participants had no information about their
partner and no means of direct communication, we also found hints that
social forces can shape minimally interactive contexts, similar to
effects observed in face-to-face interaction.

\textbf{Keywords:}
interpersonal coordination; human communication; online experiments;
social interaction
\end{abstract}

\section{Introduction}\label{introduction}

Research on the phenomenon of \emph{interpersonal coordination} focuses
on the subtle ways in which our interactions with others directly affect
our own behaviors, feelings, and thoughts. Interest in coordination
(also known as interactive alignment, interpersonal synchrony, mimicry,
and more; see Paxton, Dale, \& Richardson, 2016) has surged over the
last several decades as a framework for understanding how contact with
others shapes our cognition and behavior, with much of it focusing on
how we become more similar over time in task-oriented or friendly
contexts.

A growing perspective in this area has taken inspiration from dynamical
systems theory, conceptualizing interaction as a complex adaptive system
from which coordination arises as an emergent phenomenon according to
contextual pressures (Riley, Richardson, Shockley, \& Ramenzoni, 2011).
A fundamental principle of this dynamical systems perspective holds that
coordination should not be static across contexts nor over time.
Exploring new contexts and contextual demands---like interpersonal
conflict (Paxton \& Dale, 2013), friendly competition (Tschacher, Rees,
\& Ramseyer, 2014), or specialized task demands (Fusaroli et al.,
2012)---change coordination dynamics has become a central part of this
perspective, laying out under what conditions coordination disappears,
increases, or demonstrates complementary rather than synchronous
in-phase patterns.

There is similar interest in comparing how coordination changes across
different behavioral or cognitive systems. Under the dynamical systems
perspective, the unique pressures of a context, the resulting
coordination dynamics, and the impact of those dynamics on the
interaction may differ over time and across settings. For example, some
of the earliest work in this subset of coordination research has found
that---during task-related interaction---individuals tend to become more
similar over time across a variety of metrics (Louwerse, Dale, Bard, \&
Jeuniaux, 2012) but that specific kinds of coordination can
differentially help or hurt outcomes (Fusaroli et al., 2012).

Broadly, during tasks that are neutral (Shockley, Santana, \& Fowler,
2003), cooperative (Louwerse et al., 2012), or competitive (but not
conflict-driven; e.g., competitive games, Tschacher et al., 2014),
individuals' behavior and cognition become more similar over time. A
range of behavioral signals, both high-level (e.g., gesture; Louwerse et
al., 2012) and low-level (e.g., postural sway; Shockley et al., 2003),
become synchronized during interaction. This synchronization occurs even
when the interacting individuals are unable to see one another (Shockley
et al., 2003) or are separated in time (Richardson \& Dale, 2005).

The systematic testing of coordination across a variety of interaction
contexts is vital to charting its dynamical landscape. This methodical
exploration of different factors will eventually enable us to identify
control parameters and key factors of initial conditions that shape how
coordination emerges and how it impacts interaction outcomes. Doing so,
however, requires an expanded view of experimental paradigms: Even as we
continue to embrace more complex naturalistic interactions (e.g., Paxton
\& Dale, 2013; Tschacher et al., 2014), to fully map the interaction
space we must also develop experimental methods for analyzing
``minimally interactive contexts'' (Hale, Pan, \& Hamilton, 2015)---that
is, situations in which our interactions with others are limited in
behavioral channel, scope, or time.

Online experiment platforms and crowdsourcing can be powerful tools for
creating both fully interactive and minimally interactive paradigms. By
connecting people digitally, researchers can fully control the
experimental experience---deciding how much social information partners
will have about one another, establishing which communication channels
can be used, and potentially crafting interactive studies for groups
beyond the dyad. Crowdsourcing platforms such as Amazon Mechanical Turk
(\url{http://www.mturk.com}) have been extensively used as a means to
collect data on individuals (Buhrmester, Kwang, \& Gosling, 2011).
However, by developing real-time interactive paradigms for these
platforms, researchers interested in social behavior can study
experimentally situated social processes beyond the lab without
compromising the richness and complexity of true interactive contexts.

\section{Does Coordination Emerge in Extraordinarily Minimally
Interactive
Contexts?}\label{does-coordination-emerge-in-extraordinarily-minimally-interactive-contexts}

Here, we build on previous findings that people become coordinated
across behavioral channels even when they have very little access to
each other. Previous work has tended to preserve elements of more
typical human interaction---like speech and language---to examine how
restricting interaction can influence coordination in other behavioral
channels (e.g., gaze coordination or postural sway entrainment;
Richardson \& Dale, 2005; Shockley et al., 2003). However, understanding
the emergence and role of coordination requires us to continue to
manipulate social settings, carving out the limits of coordination to
identify the processes and constraints that create it.

To do so, we focus on task performance within a minimally interactive
context through a real-time cooperative online experiment---a nominal
game that asks players to correctly perceive and remember the length of
a line while under cognitive load. Specifically, the current study
focuses on understanding how interacting individuals become entrained in
perception and memory over time, becoming a ``line estimation system''
(cf. Dale, Richardson, \& Kirkham, 2011). This allows us to continue
mapping the course of coordination across cognitive and behavioral
systems: Building on a robust tradition on transactional memory and
collective cognitive systems (e.g., Tollefsen, Dale, \& Paxton, 2013),
we explicitly test whether low-level perception and memory processes
become more similar through contact with others.

We approach the current study with three main research questions. First,
we ask whether people become more coupled in their perceptual and memory
systems over time, despite limited perceptual and social information
about their partner. Next, we investigate whether any observed
coordination effects could simply be an artifact of the joint learning
context. Finally, we look to whether any social factors (such as rapport
and affect, which play vital roles in face-to-face interaction; e.g.,
Tschacher et al., 2014) might influence these dynamics, despite the
minimal context. We are interested to explore whether some social
influences surface as emergent effects even though the game does not
facilitate any explicit social behaviors.

\section{Method}\label{method}

All research activities were completed in compliance with oversight from
the Committee for the Protection of Human Subjects at the University of
California, Berkeley.

\subsection{Participants}\label{participants}

\begin{CodeChunk}
\begin{figure*}[h]

{\centering \includegraphics{figs/2-col-image-1} 

}

\caption[Experiment flow]{Experiment flow}\label{fig:2-col-image}
\end{figure*}
\end{CodeChunk}

Participants (\emph{n} = 148) were individually recruited from Amazon
Mechanical Turk to participate as dyads (\emph{n} = 74). Participants
were paired in the order they arrived to experiment. All participants
were over 18 years of age and were fluent English speakers
(self-reported); recruitment was restricted to participants within the
U.S. with a 95\% approval
rate.\footnote{Approval rate is a measure of MTurk worker quality, capturing how often their work is rejected by a requester. A 95\% approval rate means that only 5\% of all of their submitted work has been rejected.}

The experiment lasted an average of 11.69 minutes (range: 7.98---21.34
minutes). All participants were paid \$1.33 as base pay for finishing
the experiment and earned a bonus of up to \$2 for the entire experiment
based on their own mean accuracy over all trials (mean = \$1.80; range:
\$0.00---\$1.95). Participants were not aware of the value of their
earned bonus until after completing the experiment.

\subsection{Procedure}\label{procedure}

Data collection was run on Amazon Mechanical Turk
(\url{http://mturk.com}) using the experiment platform Dallinger
(v3.4.1; \url{http://github.com/dallinger/Dallinger}). Code for the
experiment is available on GitHub
(\url{http://github.com/thomasmorgan/joint-estimation-game}).

Each participant was individually recruited on Amazon Mechanical Turk to
play a ``Line Estimation Memory Game'' (advertisement: ``Test your
memory skills!''; see Fig. 1 for experiment flow). Upon completing
informed consent, participants were told (1) that they would be playing
a game in which they would be required to remember and recreate line
lengths; (2) that they would first complete their training trials
individually and would then play with a partner; and (3) that they would
receive a bonus based on their own accuracy on the final guess of each
test trial. Participants were given no information about their partner
other than being able to see the guess that their partner made during
test trials.

On each trial (i.e., each new stimulus set), participants were shown 3
red lines, each of a different length, and were asked to remember their
lengths.\footnote{In a pilot study, participants performed at ceiling when given only 1 line to remember and recreate. Two more lines were added to increase the memory load.}
The three lines were left-aligned within a 500x25px box and were
displayed for 2 s, followed by a blank screen for 0.5 s. Participants
were then provided with an empty 500x25px box and given 1 s to recreate
the length of the target line (\#1, \#2, or \#3). Participants made
decisions by positioning their cursor over the box at their estimate of
the rightmost extremity of the line and clicking.

During training, participants were given feedback in the form of the
true length of the target line (as a grey bar above their own guess) for
2 s. This was accompanied by a message telling the participant that they
had guessed correctly (i.e., within 4px of the true line length; ``Your
guess was correct!'') or incorrectly (``Your guess was incorrect'') or
that they had not submitted a guess within the 1-s time limit (``You
didn't respond in time'').

During testing, participants no longer received feedback their accuracy.
Instead, after both participants had submitted their first guess, they
were shown their guess placed above their partner's guess (see Fig. 1)
and were asked whether they wanted to change their own guess. Each
participant could individually change their own guess (again with a 1-s
time limit) while seeing their partner's previous guess; participants
were not informed of their partner's decision (to keep or change their
guess) until after both participants had answered (and, if needed,
changed their guess). Each trial ended when both participants were
satisfied with their guess.

Participants were informed that their final accuracy bonus would only be
calculated using their final guess. However, because they had no means
to communicate with their partner about whether each would be accepting
or changing their guesses, each participant could not have known whether
their decision to keep the guess would have been their final guess for
the trial. As a result, our statistical models use all guesses, not just
final guesses (see next section for more detail).

All dyads completed 10 training trials (alone) and 15 test trials (with
their partner). All training and test stimuli were randomly generated
for each dyad, but both participants within the dyad were given the same
stimuli. Stimuli were drawn from a uniform distribution between 1\% and
100\% (inclusive) of the total possible line length; this could have, by
chance, resulted in some relatively easier stimulus sets for some dyads,
which should be mitigated by our sample size. After participating, each
participant completed a series of questionnaires about the game on a
series of 1-10 Likert-style scales, including the perceived difficulty
of the task, how engaged they were in the task, and questions about
their own and their partner's cooperativeness and trustworthiness.

\subsection{Measures and Model
Specifications}\label{measures-and-model-specifications}

For clarity, we present the measures and model specifications together.
Each measure used in one of our three models model is defined and
written in bold the first time it is presented in this section.

\subsubsection{Model 1 Specifications: Do Partners' Perception and
Memory Couple in Minimally Interacting
Contexts?}\label{model-1-specifications-do-partners-perception-and-memory-couple-in-minimally-interacting-contexts}

Our first model tested our hypothesis that individuals' ratings would
became more similar over time. To do that, we first calculated each
participant's \textbf{error} for \emph{each guess} of \emph{each trial}.
Error was measured as a ratio relative to the total possible error on a
given target stimulus trial. That is, rather than taking a given guess's
error relative to the total line length, error was calculated as the
maximum \emph{possible} error. For example, if the target stimulus was
60 units long, participants could either under-estimate the line length
by 60 or over-estimate it by 40. As such, the maximum possible error for
that trial would be 60, and the participant's error would be calculated
relative to that maximum possible error. We chose to use normalized
error---rather than absolute error---as a measure of performance that
natively controlled for the ``possible wrong-ness'' associated with any
given line.

We then quantified perceptual and memory \textbf{coordination} (or how
similar participants' perceptual and memory systems became over time) as
the cross-correlation coefficient of participants' error.
Cross-correlation---a common measure of coordination (Paxton \& Dale,
2013)---was calculated using all guesses across all trials within a
window of +/-5 guesses. Although cross-correlation produces information
about leading and following behavior, we have no \emph{a priori}
expectations about which of the two participants would emerge as a
leader (given they have no information about their partner nor any
assigned roles). Our first-pass analyses therefore ignore any
directionality by incorporating \textbf{absolute lag}, averaging across
the correlation value for each absolute lag (i.e., leading/following in
both participants' directions).

To provide a baseline measure of \textbf{training improvement}, we
calculated the slope of each participant's normalized error over all
training trials. To account for individual differences in self-assessed
task difficulty, we used ordinal \textbf{ratings of difficulty} that
each participant gave after the task.

Our first model was a linear mixed-effects model predicting coordination
of normalized error with absolute lag and training improvement as fixed
effects, using dyad and difficulty ratings as random effects.

\subsubsection{\texorpdfstring{Model 2 Specifications: Can We Identify
Signatures of Learning \emph{and} Coordinative
Processes?}{Model 2 Specifications: Can We Identify Signatures of Learning and Coordinative Processes?}}\label{model-2-specifications-can-we-identify-signatures-of-learning-and-coordinative-processes}

During the experiment, both participants are not simply influencing one
another (as tested in Model 1)---but are also simultaneously learning to
play the game. To ensure any similarity found by Model 1 would not be
simply an artifact of both participants improving individually, we
tested the relation between participants' (1) adaptation to their
partner's perceptual estimation and memory and (2) own performance
changed over time. If participants were adapting along both avenues, we
could find evidence of these dual processes through differences in their
rates of adaptation over time.

To do this, we used the normalized error values (described above) to
derive two measures. To answer the latter point, we used each
participant's normalized error for \emph{each guess} in \emph{each
trial} as their \textbf{true error}---in other words, how much the
participant differed from the stimulus. To answer the former, we
calculated the absolute difference between both participant's true error
to obtain the \textbf{partner error} for \emph{each guess} in \emph{each
trial}---or how much the participant's guess different from their
partner's.

Because we are interested in understanding this process dynamically, we
captured participants' progress over time by creating a
\textbf{cumulative guess counter}, serving as a form of abstracted time
spent engaging with one another and the experiment. While the measure of
coordination in Model 1 presented a time-abstracted measure of
coordination across the entire experiment, this model provides a
snapshot of coordination in real time, measuring the learning and
coordinative processes from guess to guess.

For Model 2, we built a linear mixed-effects model predicting the
cumulative guess counter with each participant's true error, partner
error, the interaction term between the two, and training improvement
(described above) as fixed effects. We also included random effects for
participant and difficulty. (We did not include dyad as a random effect
in this model because the variance in the guesses occurred at the
participant level.)

\subsubsection{Model 3 Specifications: Do Social Factors Impact
Coordination in Minimally Interacting
Contexts?}\label{model-3-specifications-do-social-factors-impact-coordination-in-minimally-interacting-contexts}

To explore the role that social judgements can play even in minimally
interactive contexts, our final model considered how trust might impact
coordination. For this model, we captured a third measure of
coordination: the participant's willingness to change their guess, which
was captured by the \textbf{total number of guesses} that each
participant submitted in each trial.

Because participants individually chose whether to keep their previous
guess or submit a new one while being able to see their partner's guess,
we could expect that participants who trust their partner more would be
more likely to change their guess---especially if there were large
\textbf{absolute differences between the partner's first guesses}. Trust
was measured as each participant's self-reported Likert-style
\textbf{rating of their trust in their partner} (``How much do you feel
you trusted your partner's opinion during the experiment?'').

Model 3 was a linear-mixed effects model predicting the total number of
guesses in a trial with fixed effects for their trust in their partner
and for the difference in participants' first guess on that trial, while
controlling for trial number and how much they improved during their own
training. Model 3 also included participant and difficulty as random
effects.

\subsection{Model Implementation}\label{model-implementation}

All models were built as linear mixed-effects models in R (R Core Team,
2016) with the \texttt{lme4} package (Bates, Machler, Bolker, \& Walker,
2015), using the maximal random slope structure for each random
intercept to achieve model convergence (Barr, Levy, Scheepers, \& Tily,
2013). All main and interaction terms were centered and standardized
prior to entry in the model, allowing the model estimates to be
interpretable as effect sizes (Keith, 2005). While we do not have space
in the current paper to provide precise model specifications, we have
made our code
(\url{http://www.github.com/a-paxton/perception-memory-coordination})
and data (\url{https://osf.io/8fu7x/}) fully and freely available for
others.

\section{Results}\label{results}

\subsection{Model 1 Results: Coordinated Error over
Time}\label{model-1-results-coordinated-error-over-time}

As predicted, we found that dyads were significantly and strongly
coupled in their error ratings (\emph{\(\beta\)}=-0.43,
\emph{p}\textless{}0.0001), with no effect of training improvement
(\emph{\(\beta\)}=-0.06, \emph{p}=0.36). In other words, players were
more likely to produce lines with similar errors at the same time, even
across repeated guesses within a single trial, regardless of how
well-adapted they were to the task during training.

\subsection{\texorpdfstring{Model 2 Results: Learning \emph{and}
Coordination}{Model 2 Results: Learning and Coordination}}\label{model-2-results-learning-and-coordination}

We found that participants' rates of adaptation to their partner
significantly differed from their rates of adaptation to the game
(\emph{\(\beta\)}=-0.06, \emph{p}\textless{}0.008; see Fig. 2). In other
words, this model revealed signatures of simultaneous learning
\emph{and} coordinative processes during the game: Players became
attuned to the learning task while coordinating with one another's
cognitive processes.

Aside from the main effect of partner adaptation, no other predictors
reached statistical significance (all \emph{p}s\textgreater{}0.25).

\begin{CodeChunk}
\begin{figure}[H]

\includegraphics{figs/unnamed-chunk-1-1} \hfill{}

\caption[Difference over time in coordinative and learning processes, or the change in guess deviation from truth (in blue) and from their partner's guess (in red) over the game]{Difference over time in coordinative and learning processes, or the change in guess deviation from truth (in blue) and from their partner's guess (in red) over the game.}\label{fig:unnamed-chunk-1}
\end{figure}
\end{CodeChunk}

\subsection{Model 3 Results: Social Signals in Minimal
Contexts}\label{model-3-results-social-signals-in-minimal-contexts}

We found that greater trust in their partner predicted a small but
statistically significant \emph{increase} in the number of iterations of
guesses within a trial (\emph{\(\beta\)}=0.07,
\emph{p}\textless{}0.042), although we found no difference in the number
of guesses based solely on the difference between the partners' first
guesses (\emph{\(\beta\)}=0.01, \emph{p}=0.65). Ratings of partner trust
were normally distributed around a mean of 5.96 (SD: 2.24; range:
1--10). In other words, although Models 1 and 2 showed participants
improving and becoming more similar across trials, participants were
more willing to concede that their partner's guess was correct when the
participant trusted their partner---regardless of how similar or
different the two partners' first guesses were on that trial.

Interestingly, we found that participants took more guesses on test
trials when they improved more in their training trials
(\emph{\(\beta\)}=0.05, \emph{p}\textless{}0.028). Assuming that those
with the greatest training improvement were the most poorly performing
initial players, this suggested that poorer-performing players were more
likely to divide the cognitive labor of the task and follow the lead of
their higher-performing partner.

We also saw an effect of trial (\emph{\(\beta\)}=-0.05,
\emph{p}\textless{}0.02), indicating that people changed their guesses
fewer times per trial as the game progressed. This could be an effect of
learning (i.e., because both participants improved and became more
similar from trial to trial), experiment fatigue (e.g., if participants
simply wanted to end the game more quickly), or some combination of the
two.

\section{Discussion}\label{discussion}

Inspired by established lines of research on interpersonal coordination,
we explored how minimally interactive contexts can shape the emergence
of interpersonal dynamics. Using an online experiment that provided
participants with only one channel of information about one
another---their estimates---we found evidence of coordination of
cognitive systems despite minimal social information and context.

As expected, we found low-level perceptual and memory coordination
between players throughout the game. Congruent with findings about
postural sway (Shockley et al., 2003) and gaze (Richardson \& Dale,
2005), the present work suggests that some behavioral and cognitive
processes can become coordinated even when separated in time and space,
given some access to the relevant process in another person and a
task-based interactive context to which that process is essential.
Individual learning and performance were unable to fully account for the
players' similarity to one another.

Finally, like other work on coordination, we found that coordination was
shaped by subtle social judgements. Within coordination research,
rapport has long been upheld as one of the important predictors of
coordination (Hove \& Risen, 2009). Similarly, we found that players'
decisions to change their guesses were influenced by their self-reported
ratings of their partner's trustworthiness.

While a relatively simple paradigm, the present work contributes to the
theoretical landscape of interpersonal coordination research in several
ways. First, we continued to expand investigations of minimally
interactive contexts, an underexplored avenue in an area that often
relies on fully interactive paradigms; the present work extends our
understanding of coordination by demonstrating that interacting
individuals coordinate even when they have \emph{only} a single channel
of task-relevant sensory information available to them. Second, we
explored the degree to which even low-level properties of memory and
perception become entrained, despite this minimal
information---explicitly probing questions of memory systems posed by
transactive memory (Tollefsen et al., 2013). Finally, we applied
questions of social impacts on coordination into lower-level behavioral
and perceptual channels. Taken together, the present study found support
for hypotheses that are natural but necessary extensions of a host of
related previous work, providing explicit tests for ideas that are often
implicitly accepted by coordination researchers.

Our findings contribute to the ongoing efforts to understand the form,
function, and emergence of coordination. We were specifically interested
in pursuing three important questions around interpersonal coordination
of perception and memory: whether it emerges during minimally
interactive contexts, whether it can be distinguished from other
contemporaneous behavioral and cognitive processes, and whether it is
influenced by subtle social judgements. In addition to these theoretical
contributions, we also hope to have provided an example of the utility
of crowdsourcing platforms to investigate core principles of
interpersonal coordination and human interaction at a larger scale
without relinquishing experimental control.

\subsection{Future Directions}\label{future-directions}

Some of the questions left open by the present study may provide
interesting avenues for future work, both for better understanding some
of the effects identified here and for extending them into novel
territory.

First, we explored trustworthiness as a social construct during a task
that allowed only minimal participation between participants. The
trustworthiness measure was intentionally posed broadly, providing a
signal of the latent social information that participants constructed
when \emph{only} their partner's task-related behavior was available to
them. While we see their responses as a signal of very subtle social
forces, we readily recognize that these ratings of trustworthiness may
have been influenced by the individual's own confidence or ability---as
many social judgements are often influenced by the assessor's own
characteristics. Future work should expand on this to explore the
interplay between individual and interpersonal assessments and should
delve more deeply into understanding what might dynamics might be
influencing these minimal social judgements.

Second, given the broad participant pools available through
crowdsourcing, this work could be expanded to examine other important
questions of scale in interpersonal coordination. The majority of
research on interpersonal coordination has tended to focus on dyadic
interaction, as we did here, but many real-world social settings include
more than two people---settings which comprehensive theories of
coordination must also capture. Crowdsourcing and real-time social
experiments enable researchers to control the interaction space much
more tightly, enabling the targeted focus on specific processes across a
massive potential participant population.

\subsubsection{Acknowledgements}\label{acknowledgements}

Our thanks go to Nelle Varoquaux and to the Dallinger development team
for their assistance in debugging the experiment. This work was funded
in part by DARPA Cooperative Agreement D17AC00004 and a Moore-Sloan Data
Science Environments Fellowship to AP (through the Gordon and Betty
Moore Foundation {[}Grant GBMF3834{]} and the Alfred P. Sloan Foundation
{[}Grant 2013-10-27{]} to the University of California, Berkeley).

\section{References}\label{references}

\setlength{\parindent}{-0.1in} \setlength{\leftskip}{0.125in} \noindent

\hypertarget{refs}{}
\hypertarget{ref-barr2013random}{}
Barr, D. J., Levy, R., Scheepers, C., \& Tily, H. J. (2013). Random
effects structure in mixed-effects models: Keep it maximal.
\emph{Journal of Memory and Language}, \emph{63}(3), 255--278.

\hypertarget{ref-bates2015fitting}{}
Bates, D., Machler, M., Bolker, B., \& Walker, S. (2015). Fitting linear
mixed-effects models using lme4. \emph{Journal of Statistical Software},
\emph{67}(1), 1--48.

\hypertarget{ref-buhrmester2011amazon}{}
Buhrmester, M., Kwang, T., \& Gosling, S. D. (2011). Amazon's Mechanical
Turk: A new source of inexpensive, yet high-quality, data?, \emph{6},
3--5.

\hypertarget{ref-dale2011how}{}
Dale, R., Richardson, D. C., \& Kirkham, N. K. (2011). \emph{How two
people become a tangram recognition system}. Aarhus University, Aarhus,
Denmark.

\hypertarget{ref-fusaroli2012coming}{}
Fusaroli, R., Bahrami, B., Olsen, K., Roepstorff, A., Rees, G., Frith,
C., \& Tylén, K. (2012). Coming to terms: Quantifying the benefits of
linguistic coordination. \emph{Psychological Science}, \emph{23}(8),
931--939.

\hypertarget{ref-hale2015using}{}
Hale, J., Pan, X., \& Hamilton, A. F. de C. (2015). Using interactive
virtual characters in social neuroscience. In \emph{Virtual reality
(vr), 2015 ieee} (pp. 189--190). IEEE.

\hypertarget{ref-hove2009s}{}
Hove, M. J., \& Risen, J. L. (2009). It's all in the timing:
Interpersonal synchrony increases affiliation. \emph{Social Cognition},
\emph{27}(6), 949.

\hypertarget{ref-keith2005multiple}{}
Keith, T. (2005). \emph{Multiple regression and beyond.} Boston, MA:
Pearson.

\hypertarget{ref-louwerse2012behavior}{}
Louwerse, M. M., Dale, R., Bard, E. G., \& Jeuniaux, P. (2012). Behavior
matching in multimodal communication is synchronized. \emph{Cognitive
Science}, \emph{36}(8), 1404--1426.

\hypertarget{ref-paxton2013argument}{}
Paxton, A., \& Dale, R. (2013). Argument disrupts interpersonal
synchrony. \emph{Quarterly Journal of Experimental Psychology},
\emph{66}(11).

\hypertarget{ref-paxton2016social}{}
Paxton, A., Dale, R., \& Richardson, D. (2016). Social coordination of
verbal and nonverbal behaviors. In P. Passos, K. Davids, \& C. J. Yi
(Eds.), \emph{Interpersonal coordination and performance in social
systems} (pp. 259--273). Abington, Oxon; New York, NY: Routledge.

\hypertarget{ref-r2016r}{}
R Core Team. (2016). \emph{R: A language and environment for statistical
computing}. Vienna, Austria: R Foundation for Statistical Computing.
Retrieved from \url{https://www.R-project.org/}

\hypertarget{ref-richardson2005looking}{}
Richardson, D., \& Dale, R. (2005). Looking to understand: The coupling
between speakers' and listeners' eye movements and its relationship to
discourse comprehension. \emph{Cognitive Science}, \emph{29}(6),
1045--1060.

\hypertarget{ref-riley2011interpersonal}{}
Riley, M. A., Richardson, M., Shockley, K., \& Ramenzoni, V. C. (2011).
Interpersonal synergies. \emph{Frontiers in Psychology}, \emph{2}, 38.

\hypertarget{ref-shockley2003mutual}{}
Shockley, K., Santana, M.-V., \& Fowler, C. A. (2003). Mutual
interpersonal postural constraints are involved in cooperative
conversation. \emph{Journal of Experimental Psychology: Human Perception
and Performance}, \emph{29}(2), 326.

\hypertarget{ref-tollefsen2013alignment}{}
Tollefsen, D. P., Dale, R., \& Paxton, A. (2013). Alignment, transactive
memory, and collective cognitive systems. \emph{Review of Philosophy and
Psychology}, \emph{4}(1), 49--64.

\hypertarget{ref-tschacher2014nonverbal}{}
Tschacher, W., Rees, G. M., \& Ramseyer, F. (2014). Nonverbal synchrony
and affect in dyadic interactions. \emph{Frontiers in Psychology},
\emph{5}, 1323.

\end{document}
