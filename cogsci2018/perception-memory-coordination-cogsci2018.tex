% Template for Cogsci submission with R Markdown

% Stuff changed from original Markdown PLOS Template
\documentclass[10pt, letterpaper]{article}

\usepackage{cogsci}
\usepackage{pslatex}
\usepackage{float}
\usepackage{caption}

% amsmath package, useful for mathematical formulas
\usepackage{amsmath}

% amssymb package, useful for mathematical symbols
\usepackage{amssymb}

% hyperref package, useful for hyperlinks
\usepackage{hyperref}

% graphicx package, useful for including eps and pdf graphics
% include graphics with the command \includegraphics
\usepackage{graphicx}

% Sweave(-like)
\usepackage{fancyvrb}
\DefineVerbatimEnvironment{Sinput}{Verbatim}{fontshape=sl}
\DefineVerbatimEnvironment{Soutput}{Verbatim}{}
\DefineVerbatimEnvironment{Scode}{Verbatim}{fontshape=sl}
\newenvironment{Schunk}{}{}
\DefineVerbatimEnvironment{Code}{Verbatim}{}
\DefineVerbatimEnvironment{CodeInput}{Verbatim}{fontshape=sl}
\DefineVerbatimEnvironment{CodeOutput}{Verbatim}{}
\newenvironment{CodeChunk}{}{}

% cite package, to clean up citations in the main text. Do not remove.
\usepackage{cite}

\usepackage{color}

% Use doublespacing - comment out for single spacing
%\usepackage{setspace}
%\doublespacing


% % Text layout
% \topmargin 0.0cm
% \oddsidemargin 0.5cm
% \evensidemargin 0.5cm
% \textwidth 16cm
% \textheight 21cm

\title{Perception, Memory, and Coordination}


\author{{\large \bf Alexandra Paxton} \\ \texttt{paxton.alexandra@gmail.com} \\ Institute of Cognitive and Brain Sciences \\ Berkeley Institute for Data Science \\ University of California, Berkeley \And {\large \bf Thomas J. H. Morgan} \\ \texttt{thomas.j.h.morgan@asu.edu} \\ School of Human Evolution and Social Change \\ Arizona State University \AND {\large \bf Jordan W. Suchow} \\ \texttt{suchow@berkeley.edu} \\ Social Science Matrix \\ University of California, Berkeley \And {\large \bf Thomas L. Griffiths} \\ \texttt{tom\_griffiths@berkeley.edu} \\ Department of Psychology \\ University of California, Berkeley}

\begin{document}

\maketitle

\begin{abstract}
With cognitive scientists' increasing interest in moving outside of the
lab, recent advances in crowdsourcing platforms can help strike a
balance between the tight experimental control of lab designs and the
affordances of web-based experiments to reach beyond traditional
undergraduate subject pools. By taking advantage of new tools,
scientists interested in social cognition and behavior can create new
designs and adapt traditional ones to deliver experiments at scale.
Dallinger is one such tool, providing researchers with an open-source
experiment platform that provides end-to-end automation of the
experiment pipeline, from participant recruitment and consent to data
de-identification and participant compensation. Here we demonstrate how
Dallinger can be used to run complex experimental studies of interactive
human social behavior, as a demonstration of its potential to study
social cognition and behavior using designs drawn from across cognitive
science.

\textbf{Keywords:}
interpersonal interaction; human communication; crowdsourcing; Dallinger
\end{abstract}

\section{Introduction}\label{introduction}

\section{Method}\label{method}

All research activities were completed in compliance with oversight from
Committee for the Protection of Human Subjects at the University of
California, Berkeley.

\subsection{Participants}\label{participants}

Participants (\emph{n} = 12) were individually recruited from Amazon
Mechanical Turk to participate as dyads (\emph{n} = 6). Participants
were paired with one another according to the order in which they began
the experiment. All participants were over 18 years of age and fluent
English speakers (self-reported), located within the U.S.

The experiment lasted an average of 11.96 minutes (range: 8.13---17.66
minutes). In return for their participation, all participants were paid
\$1.33 as base pay for finishing the experiment. Each participant also
earned a bonus based of up to \$2 for the entire experiment based on
mean accuracy over all trials (mean = \$1.85; range: \$1.72---\$1.93).

\subsection{Procedure}\label{procedure}

All data collection procedures were completed through the experiment
platform Dallinger (v3.4.1;
\url{http://github.com/dallinger/Dallinger}), deployed on Amazon
Mechanical Turk (\url{http://mturk.com}). Code for the experiment is
available on GitHub
(\url{http://github.com/thomasmorgan/joint-estimation-game}), and the
resulting experiment data are available on the OSF repository for the
project (\url{https://osf.io/8fu7x/}).

Each participant was individually recruited on Amazon Mechanical Turk to
play a ``Line Estimation Game'' (advertisement: ``Test your skills!'').
Upon completing informed consent, participants were told that they would
be playing a game in which they would be required to remember and
re-create line lengths. Participants were informed that they would be
complete their training trials individually and would then begin playing
with a partner. Participants were given no information about their
partner other than the guess that their partner made; no information
about the partner's identity was shared.

In each trial, participants were shown 3 red lines (see figure;
\textbf{NB}: add figure) and were asked to remember all three of
them.\footnote{A pilot version of this study showed that participants adapted learned too quickly when given only 1 line to remember and recreate. The additional 2 lines were added to strictly increase the memory load, as opposed to adding difficulty in other ways (e.g., creating a moving stimulus).}
The 3 stimulus lines were displayed for 2 seconds then removed,
providing participants with a blank screen for 0.5 seconds. Participants
were then told which line to re-create (\#1, \#2, or \#3) and were then
given 1 second to submit their guess at how long the target line had
been. To do so, participants were given a blank box and used their
cursor to fill in the box with a blue line.

During training, participants were then shown the correct length of the
target line (as a grey bar above their own guess) for 2 seconds. This
was accompanied by a message telling the participant that they had
guessed correctly (``Your guess was correct!'') or incorrectly (``Your
guess was incorrect'') or that they had not submitted a guess within the
1-second time limit (``You didn't respond in time'').

During testing, participants' stimulus viewing, waiting, and recreation
times remained the same as during testing, but they no longer received
information about whether their guess was correct or incorrect. Instead,
after both participants had submitted their first guess, participants
were shown their guess (in blue) above their partner's guess (in green).
Both participants were then asked whether they wanted to change their
own guess or to keep the guess they had submitted. If either participant
in the dyad indicated that they wanted to change their guess, that
participant was then allowed to change their guess (again with a
1-second time limit) \emph{while} still being able to view their
partner's guess. Participants who chose to keep their previously
submitted guess was informed that their partner chose to submit a new
guess and waited for the other participant to finish. At that point,
participants were again allowed to change or keep their guess. This
process continued until both participants chose to keep their guess.

Participants were informed that their final accuracy would only be
calculated for their final guess. However, because they had no means to
communicate with their partner about whether each would be accepting or
changing their guesses, each participant could not have known whether
their decision to keep the guess would have been their final guess for
the trial.

For clarity, we will refer to each new stimulus set as a \emph{trial}
and to each submitted line length estimate within each trial as a
\emph{guess}. This means that some participants may have submitted
multiple guesses per trial. The last submitted estimate---the one by
which trial-level accuracy is calculated---will be referred to as the
\emph{final guess}.

\subsection{Analyses}\label{analyses}

\section{Results}\label{results}

\section{Discussion}\label{discussion}

\section{Conclusion}\label{conclusion}

\section{Acknowledgements}\label{acknowledgements}

Thanks go to the Dallinger development team for their assistance in
executing the experiment.

This work was funded in part by the \textbf{DALLINGER GRANT INFO GOES
HERE}.

\section{References}\label{references}

\setlength{\parindent}{-0.1in} \setlength{\leftskip}{0.125in} \noindent

\end{document}
